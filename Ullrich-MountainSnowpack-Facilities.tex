\documentclass[11pt]{article}
%\documentclass{ametsoc}
\usepackage{natbib}
\usepackage[margin=1.00in]{geometry}
\usepackage{graphicx}
\usepackage{wrapfig}
\usepackage{rotating}
\usepackage{amsmath}
\usepackage{graphicx}
\usepackage{multicol}
\usepackage{natbib}
\usepackage{wrapfig}
\usepackage{hyperref}
\usepackage{tabularx}
\usepackage{setspace}
\usepackage{comment}
\usepackage{bibentry}

\begin{document}

\setcounter{section}{2}

\section{Facilities}

This project requires computational resources for implementation and testing of software over the course of the project. The University of California, Davis has recently invested in building a campus-wide computing cluster for the advancement of scientific research in the environmental sciences, of which a significant portion was provided as part of the PI's start-up package. Consequently the PI and his research team will have high-priority access to this cluster, with over 1,500 processors, including full-time technical support from the campus computing team. A laptop for the graduate student researcher is also provided by the PI's start-up package.

Benchmark tests have been run to verify the computational capacity of the UC Davis computing cluster for VRGCM simulation. Several preliminary simulations over the western US domain using the Atmospheric Model Intercomparison Project (AMIP) protocols, prescribed sea surface temperatures and sea ice with fully coupled atmosphere and land model simulations, were also conducted on the UC Davis computing cluster.  These simulations resulted in a  throughput of 6 minutes/1 day simulated for a 28km VRGCM simulation (12 nodes) and a throughput of 9 minutes/1 day simulated for a 14km VRGCM simulation (20 nodes).  Multiple simulations will also be run in parallel to improve computational throughput.

\end{document}