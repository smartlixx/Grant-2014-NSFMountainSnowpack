\documentclass[11pt]{article}
%\documentclass{ametsoc}
\usepackage{natbib}
\usepackage[margin=1.00in]{geometry}
\usepackage{graphicx}
\usepackage{wrapfig}
\usepackage{rotating}
\usepackage{amsmath}
\usepackage{graphicx}
\usepackage{multicol}
\usepackage{natbib}
\usepackage{wrapfig}
\usepackage{hyperref}
\usepackage{tabularx}
\usepackage{setspace}
\usepackage{comment}

\begin{document}

\setcounter{section}{7}

\section{Budget Justification}
Expected Award Instrument:  Grant

\subsubsection{Inflation}
The inflation rate is assumed to be three percent per year on salaries and travel expenses.

\subsubsection{Salaries}
\textbf{Principal Investigator} -
The principal investigator [Paul Ullrich] will be reimbursed for one month of summer expenses starting in FY15, following the standard UC Davis salary track for fiscal year faculty. This rate amounts to \$8201 per month in FY15 and \$8437 per month in FY16. The PI is already funded for summer salary in FY14 via a UC Davis start-up award.
\\\\
\textbf{Graduate Student Researchers} -
One graduate student researcher [Alan Rhoades or Xingying Huang] (GSR4; nine months at 48 percent, three summer months at 100 percent) will be given a stipend each year as part of this project. The student will be from the graduate group in atmospheric science at UC Davis. This project will be conducted as part of his/her Ph.D. degree. The monthly salary rate for a GSR4 is \$3775 for July 2015 through June 2016, with inflation applied in subsequent years.

\subsubsection{Fringe Benefits}
Fringe benefit rates for those working on the project are standard UC Davis rates, as follows: Faculty summer salary (17 percent in Year 2, 18 percent in Year 3) and Graduate Student Researcher (1.3 percent).

\subsubsection{Travel and Living}
The budget includes domestic flight costs to major conferences, including the American Meteorological Society (AMS) annual meeting and American Geophysical Union (AGU) annual meeting. Specifically, the budget includes one trip per year for the PI and student, with some additional funding expected to be available from UC Davis. Travel costs and miscellaneous expenses are estimated at \$300 for the AGU conference which is traditionally held in San Francisco, CA and \$600 for the AMS conference which varies by location through the continental United States. The domestic subsidence rate is estimated to be \$180 per day (which includes hotel and per diem meal costs) plus local transportation costs of \$30 per day. Registration fees are estimated at \$600 per year.

\subsubsection{Other Direct Costs}
\textbf{Publication Costs}
Publication costs are incurred from publication of work produced by this project. The estimated cost is \$1200 per year for publication costs.
\\\\
\textbf{Software  Licenses}
Software license charges include a one time \$145 charge for the mathematics software package Maple and one \$165 / year charges for the software package Matlab. These software packages will be used by the student for data processing, intercomparison and modeling. Other software, such as Microsoft Office, is available for free via a UC Davis license.
\\\\
\textbf{Other Direct Costs:  Tuition}
The UC Davis 2015-16 estimated graduate California resident student fees are \$5905 per quarter, with no tuition paid during the summer quarter. Tuition is expected to increase by 10 percent for each academic year thereafter. This amounts to \$17,715, \$19,487 and \$21,436 for years 1, 2 and 3.

\subsubsection{Indirect Cost}
Indirect costs are charged by UC Davis on salaries, supplies, travel and hosting at a rate of 55.5 percent, 56.5 percent and 57 percent for years 1 through 3.

\subsubsection{Personnel and Work Effort}
\textbf{Paul Ullrich} - Eight percent effort

\end{document}